\chapter{Introdução}

%------------ Seção -------------
\section{Apresentação}

    % -- Ideias
    %   --- Educação física
    %    ---- atividades fisicas
    %    ---- mercado fitness
    %    ---- wearables
    %    ---- monitoramento automatizado
    %    ---- aplicativos
    %   --- Fisioterapia
    %     ---- Recuperação de membros e músculos
    %   --- Acompanhamento de músculos como parte das duas áreas citadas - atividades físicas e fisioterapia
    
    % -- Melhorias
    %   - Buscar dados estatísticos (sobre o que foi apresentado até então) que sejam de fontes válidas para citação em artigo 

    A prática de atividades físicas tem crescido exponencialmente, tornando-se um fator eminente na sociedade ao passo que é alvo de tantas atenções. Seja por motivos de saúde ou de estética, este hábito tem expandido um mercado já promissor que, anteriormente, se concentrava em um nicho de consumidores específicos cujos objetivos são ainda mais específicos, como é o caso de atletas e competidores profissionais de fisiculturismo. Porém, diante da procura de um público muito mais amplo e diversificado, e cada vez mais engajado na busca por saúde e estética, surge a necessidade de educar e entender o desenvolvimento motor dos indivíduos durante a prática dos exercícios físicos, a fim de evitar problemas com lesões à medida que também torna-se possível reduzir os riscos relacionados a uma série de doenças, principalmente àquelas de origem cardiovascular.
    
    Seja para praticantes principiantes ou avançados, há uma necessidade de monitorar os movimentos executados a fim de obter maior êxito perante os objetivos designados. Nesse cenário, a demanda pelos serviços prestados por profissionais de educação física cresce de modo desproporcional à disponibilidade ofertada pelos empregadores, dificultando um acompanhamento preciso e adequado para cada praticamente de exercícios. 
    
    Além da aplicabilidade na prática de atividades físicas, o monitoramento de sinais fisiológicos é bastante utilizado na fisioterapia, uma vez que esta ciência é de suma importância no tratamento de disfunções cinéticas funcionais tanto de órgãos quanto de sistemas. Para o caso específico no qual as lesões ocorreram ou não em decorrência de acidentes relacionados à prática de atividades físicas, a fisioterapia é um dos meios mais apropriados no treino de recuperação muscular. No entanto, como se trata de situações mais delicadas, uma vez que a lesão já existe, os serviços prestados por profissionais qualificados são considerados essenciais para o desenvolvimento das atividades de recuperação, em paralelo com o requerimento de aparelhos mais sofisticados de monitoramento, o que causa um aumento direto no custo de tratamentos fisioterapêuticos.
    
    Dados os problemas de custo e acesso existentes nas duas situações expostas até então, um dos modelos de propostas que vem ganhando espaço equivale na utilização de dispositivos \textit{wearables} ("vestíveis", numa tradução livre) que, por serem constituídos por sensores, monitoram diversos parâmetros fisiológicos do corpo humano durante a prática das mais diversas atividades musculares e, ainda, possibilitam o fornecimento de determinadas informações ou instruções específicas baseadas no processamento dos dados adquiridos.
  
    Nesse contexto, e para ambas as atividades citadas, um tipo de órgão constantemente monitorado é o músculo do tipo esquelético, que além de ser extremamente exigido em atividades físicas é o foco de recuperação da lesão em muitos casos de tratamentos fisioterapêuticos. Por consequência, existem diversas soluções para o acompanhamento dessas atividades, seja de forma mais complexa, com o emprego de aparelhos mais sofisticados e de alto custo; ou na forma de dispositivos \textit{wearables}, que possuem, de modo geral, uma proposta inicial baseada no barateamento de despesas, na praticidade e na personalização, mas que, apesar do seu grande potencial, ainda se mostram com muitas lacunas a serem preenchidas. 

    
%------------ Seção -------------
\section{Motivação}

    % Ideias
    % - Praticidade para iniciantes em exercícios físicos detectarem ou não a efetividade dos movimentos realizados
    % - Monitoramento de desempenho de atletas de alto desempenho
    % - Monitoramento de recuperação muscular em tratamentos fisioterapêuticos (ou fisioterápicos)



    A motivação do presente trabalho consiste em monitorar de forma automatizada e relativamente simples o desenvolvimento de atividades físicas, tanto para praticantes iniciantes, quanto para atletas de alto rendimento, partindo da perspectiva das necessidades apresentadas por cada um desses dois grupos. Enquanto a necessidade do primeiro grupo, referente aos iniciantes, diz respeito, sobretudo, à execução correta ou não dos movimentos, apesar de, ainda assim, não dispensar a presença de um profissional da educação física devidamente capacitado; as vantagens propostas para o segundo grupo, composto por atletas, irão além, de modo a qualificar o desempenho alcançado e servindo como um parâmetro para estabelecer as devidas alterações no treinamento que otimizarão os resultados pretendidos. 
    
    Uma outra motivação concerne em atribuir mais praticidade e menor custeamento na verificação dos movimentos realizados por pacientes que se encontram em trabalho de recuperação muscular através da fisioterapia. Atualmente, essa verificação é feita, comumente, pelo intermédio de aparelhos de alto custo e complexidade, ou, até mesmo, através de um monitoramento empírico, como, por exemplo, uma simples observação acerca da capacidade motora do paciente.
    
    
    

%------------ Seção -------------
\section{Objetivos}

    \subsection{Objetivo Geral}
    O objetivo geral do trabalho exposto está relacionado ao desenvolvimento de um sistema de verificação de efetividade na execução de movimentos corporais, através da ativação de músculos para que, desse modo, possam ser realizadas análises relativas ao estabelecimento dos melhores movimentos para alcançar determinados objetivos, tais quais hipertrofia muscular ou recuperação de lesões via fisioterapia.
    
    
    \subsection{Objetivos Específicos}
        \begin{itemize}
        \item Reunir informações sobre a relação entre movimentos em membros do corpo e utilização de músculos relacionados a tais membros;
        \item Desenvolver dispositivo que detecte a movimentação de um membro do corpo humano e transmita essa informação para um dispositivo com sistema operacional Android;
        \item Desenvolver dispositivo que detecte nível de utilização de um músculo e transmita essa informação para um dispositivo com sistema operacional Android;
        \item Desenvolver aplicativo Android que capte as informações enviadas pelos dispositivos citados anteriormente e determine a efetividade do movimento feito;
        \item Testar resultados desse sistema como um todo e comparar conclusões do aplicativo com conclusões encontradas na literatura.
        \end{itemize}
        %Possibilidades: verificar fadiga do músculo, fazer recomendações de quantas repetições deve-se fazer, recomendar treinos, baseando-se no fato de que um exercício já não tem mais eficácia para o usuário, caso o mesmo tenha executado este diversas vezes, etc.


%------------ Seção -------------

\section{Metodologia e Cronograma de Atividades}

    Para alcançar os objetivos pretendidos, este trabalho se inicia com o estudo da utilização dos possíveis sensores e outros dispositivos afins a compor o sistema embarcado de captação dos sinais desejados: movimentação corporal e utilização do músculo.
    
    Em seguida, é avaliada a melhor forma, em termos de eficiência e custo, de transmissão dos sinais captados pelos sensores que deverão enviar as informações coletadas para um dispositivo móvel com sistema operacional Android. 
    
    Por último, é desenvolvido um aplicativo móvel para dispositivos Android que faça a devida leitura do sinal através do protocolo de transmissão escolhido, para então processa-lo, comparando os parâmetros movimento e nível de utilização do músculo; dando um retorno ao usuário, a depender das variáveis de uso escolhidas (fisioterapia, atividade física iniciante ou atividade física avançada), acerca da porcentagem de eficácia do músculo utilizado para determinado fim, além de prover gráficos a compor o histórico do usuário, possibilitando o adicional do acompanhamento e avaliação otimizados por parte dos profissionais de cada área. 



%------------ Seção -------------

\section{Organização do Trabalho}

    Para um melhor entendimento do trabalho realizado, esta obra foi dividida nos seguintes capítulos:
    
    \begin{itemize}
        \item {Capítulo 1 - Introdução}: aborda as informações básicas necessárias para o entendimento do trabalho, assim como as motivações para a realização do mesmo e sua relevância perante trabalhos existentes.
        \item {Capítulo 2 - Fundamentação Teórica}: expõe os conceitos teóricos e fundamentos para o entendimento da execução do trabalho.
        \item {Capítulo 3 - Trabalhos Relacionados}: analisa a literatura e produtos relacionados ao trabalho proposto, visando ter um alicerce sólido para o desenvolvimento da proposta, assim como para evitar a repetição de experimentos já bem sucedidos e contribuir com o estado da arte.
        \item {Capítulo 4 - Proposta}: são detalhados os procedimentos necessários para o desenvolvimento do trabalho.
        \item {Capítulo 5 - Resultados}: expõe as informações resultantes do da análise feita no aplicativo de forma organizada e de fácil entendimento
        \item {Capítulo 6 - Conclusão}: compara os resultados obtidos no experimento com os resultados obtidos na literatura, sendo esta feita através da interpretação humana dos resultados, a fim de verificar a eficácia da automação nesse processo de acompanhamento de atividade física. Além disso, serão feitas propostas de continuidade e melhorias deste trabalho.
    \end{itemize}
